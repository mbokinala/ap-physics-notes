\documentclass[titlepage]{article}

% Language setting
% Replace `english' with e.g. `spanish' to change the document language
\usepackage[english]{babel}

\usepackage{fancyhdr}

% Set page size and margins
% Replace `letterpaper' with`a4paper' for UK/EU standard size
\usepackage[letterpaper,top=2cm,bottom=2cm,left=3cm,right=3cm,marginparwidth=1.75cm]{geometry}

% Useful packages
\usepackage{amsmath}
\usepackage{graphicx}
\usepackage{physics}
\usepackage[colorlinks=true, allcolors=blue]{hyperref}

\usepackage[parfill]{parskip}

% tables
\usepackage{geometry}
\usepackage{makecell}
\usepackage{multicol}
\usepackage{float}
\usepackage{lscape}
\usepackage{amsmath}
\usepackage[font=small,labelfont=bf,justification=centering]{caption}

\setcellgapes{10pt}

\captionsetup[table]{justification=raggedright,singlelinecheck=off}

% \setlength{\parindent}{0pt}

\setcounter{tocdepth}{4}
\setcounter{secnumdepth}{4}

\usepackage{gensymb}

\usepackage{tgpagella}
\renewcommand{\familydefault}{\sfdefault}

\title{AP Physics C: Mechanics - Notes}
\author{Manav Bokinala}
\date{2021 - 2022}

\pagestyle{fancy}
\fancyhf{}
\fancyhead[LE,RO]{Manav Bokinala}
\fancyhead[RE,LO]{AP Physics C: Mechanics - Notes}
\fancyfoot[LE,RO]{\thepage}

\begin{document}
\maketitle

\newpage
\tableofcontents
\newpage

\section{Vectors and Motion in 2 Dimensions}
Unit vector: length of 1, no units (used for direction)

\begin{itemize}
    \item Denoted with "hat"
    \item $\hat{x}$, $\hat{y}$, $\hat{z}$ - usually denoted as $\hat{i}$, $\hat{j}$, $\hat{k}$
\end{itemize}

For a 2D velocity vector:
\[ \vec{v} = v_x \hat{i} + v_y \hat{j} \]

\subsection{Polar Notation}
\[ \vec{v} = \sqrt{{v_x}^2 + {v_y}^2} \text{ at } \theta \degree \]

$\theta$ is angle relative to \emph{positive x-axis}

\subsection{Vector Addition}
\begin{align*}
    \vec{A}           & = A_x \hat{i} + A_y \hat{j}                 \\
    \vec{B}           & = B_x \hat{i} + B_y \hat{j}                 \\
    \vec{A} + \vec{B} & = (A_x + B_x) \hat{i} + (A_y + B_y) \hat{j}
\end{align*}

\subsection{Scalar Multiplication}
scalar $\cdot$ vector = vector

\begin{align*}
     & \vec{A} = A_x \hat{i} + A_y \hat{j}    \\
     & 2\vec{A} = 2A_x \hat{i} + 2A_y \hat{j}
\end{align*}

\subsection{Vector Multiplication}
vector $\cross$ vector\\

scalar/dot product: $\vec{A} \cdot \vec{B}$
\begin{itemize}
    \item results in scalar
\end{itemize}

vector/cross product: $\vec{A} \cross \vec{B}$
\begin{itemize}
    \item results in vector
\end{itemize}

\subsection{Dot Product}
\[
    \vec{A} \cdot \vec{B} = \abs{A} \abs{B} \cos \theta
\]

\begin{itemize}
    \item $\theta$ is the angle between the two vectors
\end{itemize}

Or,

\[ \vec{A} \cdot \vec{B} = A_xB_x + A_yBy \]

\subsubsection{Dot Product Properties}
\begin{align*}
    \vec{A} \cdot \vec{B} & = \vec{B} \cdot \vec{A}                               \\
    \vec{A} \cdot \vec{B} & = 0 \text{ if $\vec{A}$ and $\vec{B}$ are orthogonal}
\end{align*}

\subsection{Cross Products}
\begin{align*}
    \text{If } \vec{C} & = \vec{A} \cross \vec{B},     \\
    \abs{C}            & = \abs{A} \abs{B} \sin \theta
\end{align*}

the cross product is \textbf{orthogonal} to both original vectors (uses 3rd dimension)

\subsubsection{Cross Product Properties}

\begin{align*}
     & \vec{A} \cross \vec{B} \neq \vec{B} \cross \vec{A} \\
     & \vec{A} \cross \vec{B} = - \vec{B} \cross \vec{A}
\end{align*}

right hand rule - cross product goes "into" or "out of" page

\section{Drag Force}
\begin{itemize}
    \item resistive force by \textbf{\emph{fluids}} on \textbf{\emph{moving objects}}
    \item \textbf{\emph{opposes}} direction of motion
\end{itemize}

\subsection{\texorpdfstring{$v^2$}{v\^{}2} model}
\[ F_{drag} = - \frac{1}{2} C \rho A v^2 \]

\begin{itemize}
    \item $C$ - drag coefficient
    \item $\rho$ - density of fluid - greek letter "rho"
    \item $A$ - cross-sectional area of object
    \item $v$ - velocity of object
    \item negative sign just indicates direction opposite to motion
\end{itemize}
\ \\
Sometimes, the coefficients $\frac{1}{2}C \rho A v^2$ are grouped into one coefficient ($b$ or $k$)

\begin{itemize}
    \item So, $F_{drag} = -bv^2$
    \item also called the drag coefficient
    \item specific to object and location
    \item calculated experimentally
\end{itemize}

\subsection{\texorpdfstring{$v$}{v} model}
for \textbf{\emph{small objects}} moving at \textbf{\emph{slow speeds}},
\[F_{drag} = -bv\]

\subsection{Terminal Velocity}
For an object in free fall, terminal velocity occurs when $\abs{\vec{F}_{drag}} = \abs{\vec{F}_{gravity}}$

\begin{align*}
    F_{net} & = F_g - F_{drag}                             \\
    0       & = mg - bv_t                                  \\
    v_t     & = \frac{mg}{b} \text{ ($v$ model)}           \\
    v_t     & = \sqrt {\frac{mg}{b}} \text{ ($v^2$ model)}
\end{align*}

\section{Differential Equations}
Differential equations define relationship between a function and one or more \emph{derivatives} of that function

Drag force is modeled with a differential equation:

\begin{align*}
    F_{drag}        & = bv \\
    ma              & = bv \\
    % \frac{d}{dt} \left[ ma \right] &= \left[ bv \right] \\
    m \frac{dv}{dt} & = bv
\end{align*}

\subsection{Integrating Differential Equations}
\begin{align*}
    m \frac{dv}{dt}     & = bv               &  & \text{rewrite equation to see dependent and independent variables} \\
    \frac{1}{v} dv      & = \frac{b}{m} dt   &  & \text{separate each variable to one side}                          \\
    \int \frac{1}{v} dv & = \int \frac{b}{m} &  & \text{integrate}
\end{align*}

\section{Circular Motion}
Centripetal force ($F_c$ or $\sum F_c$) isn't really a force - it is the \textbf{net} force along the centripetal axis (the axis going through a point on the circular path and the center of said circle).
\begin{itemize}
    \item could be weight, tension, friction, etc.
    \item is always orthogonal (perpendicular) to velocity
    \item depends on mass and tangential velocity of object and radius of circular path
\end{itemize}

\subsection{Uniform Circular Motion}
Uniform circular motion occurs when an object moving in a circle with a fixed radius has a constant tangential (and therefore angular) speed

\textbf{frequency} ($f$) is measured in Hertz {Hz}\\
\textbf{period} ($T$) is measured in seconds (s)

\subsection{Velocity}
\[v_t = \frac{\Delta x}{T} = \frac{2\pi r}{T} = 2\pi rf\]

\subsection{Acceleration}
\[a_c = \frac{v^2}{r} = \frac{4 \pi^2 r}{T^2}\]

Centripetal acceleration always points toward the center of the circular path
\begin{itemize}
    \item does not change \emph{speed} of object; only changes \emph{direction}
\end{itemize}

\subsection{Force}
Centripetal force is the sum of all forces on the centripetal axis

\begin{align*}
    \sum F   & = ma             \\
    \sum F_c & = ma_c           \\
    \sum F_c & = \frac{mv^2}{r}
\end{align*}

points toward the center of the circular path

\section{Rotational Motion}

\textbf{rigid bodies}: objects that do not change shape (deform)\\
\textbf{translational motion}: motion of the \emph{center of mass} of the object\\
\textbf{rotational motion}: rotation around a fixed point (not always the center of mass)
\ \\
\begin{table}[H]
    \makegapedcells
    \begin{tabular}{c|c}
        \textbf{Circular Motion}                    & \textbf{Rotational Motion}                                \\
        \hline

        circular path                               & spins about axis of rotation                              \\
        all parts of object move with same velocity & different points on the object travel at different speeds \\
    \end{tabular}
\end{table}

\subsection{Variables}
\begin{itemize}
    \item $R$ - distance from point on object to axis of rotation
    \item $\theta$ - angular displacement - \emph{measured in radians}
    \item $S$ - Arc Length
\end{itemize}
\[S = R\theta\]

\subsection{Angular Velocity}
rate of change of $\theta$
\[ \omega = \frac{\Delta \theta}{t} = 2 \pi f \]

\begin{itemize}
    \item $\omega$ - angular velocity - \emph{measured in radians/second}
\end{itemize}
\ \\
all points on a rigid body have the same angular velocity

$\omega$ does \textbf{\emph{not}} point in the direction of rotation
\begin{itemize}
    \item use the right hand rule to determine the direction of $omega$ - always orthogonal to the rotational plane
          \begin{itemize}
              \item usually described as "into page" or "out of page"
              \item sometimes described as "clockwise" or "counterclockwise"
          \end{itemize}
\end{itemize}

\subsection{Angular Acceleration}
rate of change of $omega$
\[ \alpha = \frac{\Delta \omega}{t} \]
\begin{itemize}
    \item $\alpha$ - angular acceleration - \emph{measured in radians/second\textsuperscript{2}}
\end{itemize}
\ \\
$\alpha$ does \textbf{\emph{not}} point in the direction of rotation - same as angular velocity ($\omega$)
\begin{itemize}
    \item use right hand rule just like angular velocity to determine direction of $\alpha$
\end{itemize}

\subsection{Kinematic Equations}

\begin{table}[H]
    \makegapedcells
    \begin{tabular}{c|c}
        \textbf{Translational Kinematics}    & \textbf{Rotational Kinematics}                         \\
        \hline
        $\bar{v} = \dfrac{\Delta x}{t}$      & $\bar{\omega} = \dfrac{\Delta \theta}{t}$              \\
        $\bar{v} = \dfrac{v_1 + v_2}{2}$     & $\bar{\omega} = \dfrac{\omega_1 + \omega_2}{2}$        \\
        $\bar{a} = \dfrac{v - v_0}{t}$       & $\bar{\alpha} = \dfrac{\omega - \omega_0}{t}$          \\
        $\Delta x = v_0 t + \frac{1}{2}at^2$ & $\Delta \theta = \omega_0 t + \frac{1}{2} \alpha t^2$  \\
        ${v_{f}}^2 = {v_0}^2 + 2a\Delta x$   & ${\omega_{f}}^2 = {\omega_0}^2 + 2\alpha\Delta \theta$ \\
    \end{tabular}
\end{table}

\section{Center of Mass (Discrete Objects)}
\begin{itemize}
    \item The point at which all \textbf{mass} of an object is though to be concentrated
    \item also thought of as "average location of mass"
    \item can be determined experimentally or mathematically
    \item the center of mass of \emph{all objects} moves like a point particle \emph{even if the object is rotating}
    \item could be inside or outside the object
    \item the geometric center of an object is \emph{not} always its center of mass - only when the object has a uniform density
    \item unless specified, objects have a uniform density
\end{itemize}

\subsection{Position of the Center of Mass}
\[x_{cm} = \frac{1}{M}\sum m_i x_i\]

\begin{itemize}
    \item $x_cm$ = position of the center of mass of the systeme
    \item $M$ - total mass of the systeme
    \item $m_i$ - mass of the $i$\textsuperscript{th} particle
    \item $v_i$ - psoition of the $i$\textsuperscript{th} particle
\end{itemize}

A continuous object can be broken down into symmetric discrete pieces to find its center of mass

\subsection{Velocity of the Center of Mass}
\[v_cm = \frac{1}{M}\sum m_i v_i\]

\begin{itemize}
    \item $v_i$ - velocity of the $i$\textsuperscript{th} particle
\end{itemize}

\subsection{Acceleration of the Center of Mass}
Newton's third law ($F = ma$) actually refers to the center of mass of a system or object:
\[Ma_{cm} = \sum F_{external}\]

\section{Center of Mass (Continuous Objects)}
Not all continuous objects have a uniform mass density
\begin{itemize}
    \item Mass density usually varies withlocation
\end{itemize}

\[x_{cm} = \frac{1}{M} \int x \ dm\]
\begin{itemize}
    \item $x$ - position of particle
    \item $dm$ - infinitely small mass
\end{itemize}
position is given as a function of mass\\
integrate along the length of the object

\subsection{Position of Center of Mass}
\begin{enumerate}
    \item Define mass density function: \[\lambda = \frac{dm}{dx}\]
    \item Separate variables: \[dm = \lambda(x) dx\]
    \item Integrate along length of object to find total mass of object:
    \item Use integral form of center of mass formula (see above) to solve for the position of the center of mass
          \begin{itemize}
              \item use total mass calculated in step 4
          \end{itemize}
\end{enumerate}

\section{Rotational Intertia}
Rotational inertia (a.k.a \textbf{moment of inertia}): ability of an object to resist changes in its rotational motion\\ \\
symbol: $I$\\
scalar quantity\\
units: kg $\cdot$ m$^2$

\subsection{Discrete Objects}
\begin{equation*}
    I = \sum m_i R_i^2
\end{equation*}

\begin{itemize}
    \item $m_i$ - mass of object
    \item $R$ - distance between axis of rotation and the object
\end{itemize}

The same object can have different rotational inertias depending on the location of the axis of rotation

\subsection{Continuous Objects}
\begin{equation*}
    I = \int R^2 \ dm
\end{equation*}

\begin{table}[H]
    \centering
    \makegapedcells

    \begin{tabular}{c|c}
        \textbf{Object and Axis of Rotation} & \textbf{Rotational Inertia} \\
        \hline
        Thin rod, about center               & $\frac{1}{12}ML^2$          \\
        Thin rod, about end                  & $\frac{1}{3}ML^2$           \\
        Cylinder or disk, about center       & $\frac{1}{2} MR^2$          \\
        Cylindrical hoop, about edge         & $MR^2$
    \end{tabular}
\end{table}

\subsection{Parallel Axis Theorem}
If the moment of inertia about the center of mass ($I_{CoM}$) is known, the moment of inertia about any parallel axis of rotation can be found using the formula:
\begin{equation*}
    I_{parallel} = I_{CoM} + Md^2
\end{equation*}

\begin{itemize}
    \item $M$ - mass of the objects
    \item $d$ distance between the axes
\end{itemize}

\subsection{Systems of Objects}
For a system of discrete and/or continous objects:
\begin{equation*}
    I_{system} = I_1 + I_2 + I_3 + \cdots + I_n
\end{equation*}

\section{Torque}
\textbf{Torque} is the ability of a force to make an object rotate - a "twisting force"

symbol: $\tau$ ("tau") \\
units: $N \cdot m$ \\
vector quantity

Magnitude depends on
\begin{itemize}
    \item size of the force
    \item direction
    \item location at which the force is applied
\end{itemize}

torque is given as the cross product $R \cross F$:
\begin{equation*}
    \tau = R \cross F = RF \sin \theta
\end{equation*}

\begin{itemize}
    \item $R$ - distance from the axis of rotation to the applied force
    \item $F$ - applied force
    \item $\theta$ - angle between $R$ and $F$ (when drawn tip-to-tail)
\end{itemize}

\subsection{Direction of Torque}
Use the right hand rule to find the direction of torque:
\begin{itemize}
    \item point index finger in direction from axis of rotation to applied force
    \item point middle finger in direction of applied forces
    \item point thumb out - this is the direction of the torque
\end{itemize}

\subsection{Torque in Newton's Second Law}
\begin{align*}
    \tau_{net} & = R \cross F                                            \\
               & = R F_{net}                                             \\
               & = Rma         &  & \text{Newton's 2nd Law, } F = ma     \\
               & = Rm \alpha R &  & a = \alpha R                         \\
               & = I \alpha    &  & \text{rotational inertia, } I = mR^2
\end{align*}

\section{Rolling}
Friction causes a torque that makes objects roll. Rolling objects experience both \textbf{rotational} and \textbf{translational} motion.

Objects can roll with or without \textbf{slipping}

\subsection{Rolling Without Slipping}
When an object rolls without slipping,
\begin{align*}
    v_{cm} = \omega R \\
    a_cm = \alpha R
\end{align*}

\begin{itemize}
    \item $v_{cm}$ and $a_{cm}$ are the velocities and accelerations of the center of mass, respectively
    \item $\omega$ and $\alpha$ are the angular velocity and acceleration of the object, respectively
    \item $R$ is the distance from the axis of rotation to the point of contact with the ground
\end{itemize}

Rolling without slipping problems are solved by relating the rotational and translational motions of an object using the equation $a = R\alpha$

\subsection{Rolling With Slipping}
When an object rolls without slipping,
\begin{align*}
    v_{cm} \neq \omega R \\
    a_{cm} \neq \alpha R
\end{align*}

Rolling without slipping problems are solved by figuring out \textbf{when} $v = R \omega$ (the object enters into rolling without slipping)

\subsubsection{Two Types of Rolling With Slipping}
\begin{enumerate}
    \item $v_{cm} > \omega R$ - object is sliding around ground at some points (e.g. bowling ball sliding and rolling slightly before it 'catches grip' and rolls at full speed)
    \item $v_{cm} < \omega R$ - object is spinning more than one revolution in the time it takes to advance by one circumference (e.g. car doing burnouts or losing traction)
\end{enumerate}

\section{Work and Energy}
\begin{itemize}
    \item energy is a scalar quantity (has no direction), but can be positive or negative'
    \item energy is always conserved in a closed system
\end{itemize}

\subsection{Work}
Work is the result of a force being applied to move an object across a displacement - measured in $N \cdot m$
\begin{equation*}
    W = F \cdot \Delta x = F \Delta x \cos \theta
\end{equation*}
\begin{itemize}
    \item work is the \textbf{cross product} of force and displacement (vector quantity)
    \item $\theta$ is the angle between $F$ and $\Delta x$ when aligned tail to tail
\end{itemize}

\subsubsection{Work-Energy Theorem}
Work changes the energy of an object
\begin{equation*}
    \Sigma W = \Delta KE
\end{equation*}
\begin{itemize}
    \item net work is equal to the change in kinetic energy of an object
\end{itemize}

\subsection{Energy}
Energy is measured in Joules ($J$) - equal to ($N \cdot m$)
\subsubsection{Types of Mechanical Energy}

\textbf{Kinetic Energy}
\begin{equation*}
    KE = \frac{1}{2}mv^2
\end{equation*}

\textbf{Rotational Kinetic Energy}
\begin{equation*}
    RKE = \frac{1}{2} I \omega ^ 2
\end{equation*}

\textbf{Potential Energy} \\
\emph{Gravitational}
\begin{equation*}
    GPE = mgh
\end{equation*}

\emph{Elastic}
\begin{equation*}
    EPE = \frac{1}{2}k (\Delta s)^2
\end{equation*}
\begin{itemize}
    \item $k$ - spring constant
    \item $\Delta s$ - displacement of the spring (compression or extension)
\end{itemize}

\subsection{Power}
power is energy delivered over a period of time

\begin{equation*}
    P = \frac{\Delta E}{\Delta t} = \frac{W}{\Delta t} = \frac{F \Delta x}{\Delta t} = Fv
\end{equation*}

\subsection{Potential Energy Curves}
graph of potential energy vs position

For \textbf{conservative forces},
\begin{equation*}
    F(x) = -\frac{d}{dx} U(x)
\end{equation*}

\subsubsection{Equilibrium Points}
Occur where the slope (equal to $F_{net}$) is zero


\section{Impulse and Momentum}
\textbf{impulse} is a force applied for a period of time

symbol: $J$ \\
units: $N \cdot s$\\
vector quantity - same direction as force

\begin{align*}
    J & = F \Delta t               \\
    J & = \int_{t_0}^{t_1} F(t) dt
\end{align*}

\textbf{momentum} is the amount of motion of an object

symbol: $p$\\
units: $kg \cdot m/s$\\
vector quantity

\begin{equation*}
    p = mv
\end{equation*}

\subsection{Conservation of Momentum}
Impulse is a change in momentum

Where A and B are two objects in a system upon which no outside forces act,
\begin{align*}
    J          & = \Delta p       \\
    \Delta p_B & = -\Delta p_A    \\
    \Delta p_A & + \Delta p_B = 0
\end{align*}

When no outside forces act on a system, momentum is conserved (the net change in momentum is 0)

\subsection{Collision Types}
\subsubsection{Elastic Collisions}
\begin{itemize}
    \item energy is consevered
    \item objects bounce off each other
\end{itemize}

\subsubsection{Inelastic Collisions}
\begin{itemize}
    \item energy is NOT conserved
    \item objects bounce off each other
\end{itemize}

\subsubsection{Perfectly Inelastic Collisions}
\begin{itemize}
    \item energy is NOT conserved
    \item objects stick to each other
\end{itemize}

\subsection{Useful Equations}
\subsubsection{Head-on Elastic Collision}
\begin{equation*}
    v_{1_0} + v_{1_f} = v_{2_0} + v_{2_f}
\end{equation*}

\subsubsection{Perfectly Inelastic Ballistic Collission}
When a projectile strikes a block on a pendulum,
\begin{equation*}
    v_0 = \frac{m + M}{m}\sqrt{2gh}
\end{equation*}
where
\begin{itemize}
    \item $v_0$ - initial velocity of the bullet
    \item $m$ - mass of the projectile
    \item $M$ - mass of the block
    \item $h$ - max height of the pendulum
\end{itemize}

\subsection{Calculus with Momentum and Impulse}
\begin{equation*}
    \int F(t) dt = J
\end{equation*}

\section{Angular Impulse and Momentum}
\subsection{Angular Momentum}
\textbf{angular momentum} is the amount of angular motion an obect has

symbol: $L$\\
units: $kg \cdot m^2 / s$\\
"pseudo-vector" quantity (like torque)

\begin{equation*}
    L = I \omega = \vec{R} \cross \vec{p} = Rp \sin \theta
\end{equation*}

\begin{itemize}
    \item $I$ - moment of inertia
    \item $\omega$ - angular velocity
    \item $\vec{R}$ - vector from aribtrary center of rotation to object
    \item $\vec{p}$ - linear momentum vector
    \item $\theta$ - angle between $R$ and $p$ (tip to tail)
\end{itemize}

NOTE: because any point can be chosen as the center of rotation, point particles can have an angular momentum

\subsection{Angular Impulse}
\textbf{angular impulse} is a torque applied for a period of time

symbol: $\Delta L$\\
units: $N \cdot m \cdot s$\\
"pseudo-vector" quantity
\begin{equation*}
    \Delta L = \tau_{net} \Delta t
\end{equation*}

\subsubsection{Newton's Third Law from Angular Impulse}
\begin{equation*}
    \tau_{net} = \frac{dL}{dt} = I\alpha
\end{equation*}

\section{Simple Harmonic Motion}
\begin{itemize}
    \item \textbf{periodic motion} is cyclic motion that repeats the same path in the same amount of time
    \item the \textbf{equilibrium position} is the position where the net force on the object is 0
    \item the \textbf{restoring force} is the force that acts opposite to displacement to bring the object back to the equilibrium position
    \item \textbf{displacement} is the position relative to the equilibrium position (measured in meters)
    \item \textbf{amplitude} is the maximum displacement of an object in periodic motion
    \item a \textbf{cycle} is a complete back and forth swing of the pendulum
    \item the \textbf{period} is the time it takes to complete one cycle
    \item the \textbf{frequency} is the number of cycles completed in a second (inverse of period)
\end{itemize}

\textbf{simple harmonic motion} occurs when the the restoring force is in the \emph{opposite direction} of, and has a magnitude \emph{proportional to} the displacement

All simple harmonic oscillators have an \textbf{angular velocity}. The period of the oscillator is given by
\begin{equation*}
    T = 2 \pi \sqrt{\frac{1}{\omega ^ 2}}
\end{equation*}

For all simple harmonic oscillators, 
\begin{equation*}
    a = \frac{d^2 x}{dt^2} = - \omega^2 x
\end{equation*}

\subsection{Simple Pendulums}
For a pendulum, the restoring force is gravity and $x = \theta L$

The restoring force of a pendulum is given by
\begin{equation*}
    F_{r} = -mg \sin \theta
\end{equation*}

where $\theta$ is the angular displacement of the Pendulum

For small values of $\theta$ (less than 20\textdegree), $\sin \theta \approx \theta$, so
\begin{equation*}
    F_r = -mg \theta
\end{equation*}

For a pendulum,
\begin{align*}
    F                  & = -mg \theta       \\
    ma                 & = -\frac{mg}{L}{x} \\
    \frac{d^2 x}{dt^2} & = -\frac{g}{L}x
\end{align*}

There are three possible solutions to the differential equation above, depending on the starting position of the pendulum

\begin{table}[H]
    \centering
    \makegapedcells
    \begin{tabular}{c|c}
        {initial position} & \textbf{$x(t)$}                   \\
        \hline
        $x_0 = 0$          & $x(t) = A(\sin{\omega t})$        \\
        $x = A$            & $x(t) = A(\cos{\omega t})$        \\
        $x = \theta_i L$   & $x(t) = A(\cos{\omega t + \phi})$ \\
    \end{tabular}
\end{table}

For a pendulum, the angular velocity is given by

\begin{equation*}
    \omega^2 = -\frac{g}{L}
\end{equation*}

\subsection{Spring Mass Systems}
The period of a spring mass system is given by
\begin{equation*}
    T = 2 \pi \sqrt{\frac{m}{k}}
\end{equation*}

The angular velocity is given by
\begin{equation*}
    \omega ^ 2 = \frac{k}{m}
\end{equation*}

\subsubsection{Vertical Springs}
If a spring stretches a distance of $\Delta L$ from its unstretched length when a mass is added to it,
\begin{equation*}
    \Delta L = \frac{mg}{k}
\end{equation*}

\subsubsection{Useful Equations for Spring Mass Systems}
\begin{align*}
    v_{max} &= A \sqrt{\frac{k}{m}} \\
    v(x) &= \pm v_{max} \sqrt{1 - \frac{x^2}{A^2}}
\end{align*}

\subsubsection{Multi Spring Systems}
When multiple springs are connected, they can be thought of as a single \textbf{equivalent spring} with a unique spring constant.

\paragraph{Springs in Parallel}
Springs are considered to be connected \textbf{in parallel} if they are \emph{both connected to an object} and \emph{not connected to each other}. Springs in parallel act as a single \textbf{more effective} spring. For springs in parallel,
\begin{equation*}
    k_{equiv} = k_1 + k_2 + \cdots
\end{equation*}

\paragraph{Springs in Series}
Springs are considered to be connected \textbf{in series} if they are \emph{connected to each other} and \emph{only one of the springs is connected to an object}. Springs in series act as a single \textbf{less effective} spring. For springs in series,
\begin{equation*}
    \frac{1}{k_{equiv}} = \frac{1}{k_1} + \frac{1}{k_2} + \cdots
\end{equation*}

\subsection{Physical Pendulums}
A physical pendulum is a \emph{rigid} extended body that rotates cyclically about a point

For a physical pendulum,
\begin{equation*}
    \frac{d^2 \theta}{dt^2} = \frac{mg R_{cm}}{I}
\end{equation*}
\begin{itemize}
    \item $R_{cm}$ is the distance from the center of mass of the pendulum to the axis of rotation
    \item $I$ is the rotational inertia of the pendulum
\end{itemize}

The angular velocity of a physical pendulum is given by
\begin{equation*}
    \omega ^ 2 = \frac{mg R_{cm}}{I}
\end{equation*}

The period of a physical pendulum is given by
\begin{equation*}
    T = 2 \pi \sqrt{\frac{I}{mg R_{cm}}}
\end{equation*}

\subsection{Energy Transfer in Simple Harmonic Motion}
The total mechanical energy of a simple harmonic oscillator is always constant

At \emph{equilibrium}, the kinetic energy (and therefore velocity) is at a \emph{maximum} and the potential energy is at a \emph{minimum} (therefore net force and acceleration are 0)

At the \emph{amplitude}, the kinetic energy (and therefore velocity) is 0 and the potential energy (and therefore net force and acceleration) is at a \emph{maximum}

\subsection{Resonance}
When a force is applied to an object with a frequency close to the natural frequency (aka resonant frequency) of the object, the amplitude of the oscillation (and therefore the total energy in the system) increases.

\subsection{Summary of Simple Harmonic Oscillator Angular Velocities and Periods}

\begin{table}[H]
    \centering
    \makegapedcells
    \begin{tabular}{c|c|c}
        {oscillator type} & $\omega^2$ & $T$                   \\
        \hline
        {simple pendulum} & $\omega^2 = \dfrac{g}{L}$ & $T = 2 \pi \sqrt{\dfrac{L}{g}}$ \\
        {spring-mass system} & $\omega^2 = \dfrac{k}{m}$ & $T = 2 \pi \sqrt{\dfrac{m}{k}}$ \\
        {physical pendulum} & $\omega^2 = \dfrac{mg R_{cm}}{I}$ & $T = 2 \pi \sqrt{\dfrac{I}{mg R_{cm}}}$
    \end{tabular}
\end{table}
\end{document}